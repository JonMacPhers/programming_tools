%% LyX 2.2.2 created this file.  For more info, see http://www.lyx.org/.
%% Do not edit unless you really know what you are doing.
\documentclass[12pt,english]{article}
\renewcommand{\familydefault}{\sfdefault}
\usepackage[latin9]{inputenc}
\usepackage{geometry}
\geometry{verbose,tmargin=1in,bmargin=1in,lmargin=1cm,rmargin=1in}
\usepackage{url}
\usepackage{amsmath}

\makeatletter
%%%%%%%%%%%%%%%%%%%%%%%%%%%%%% User specified LaTeX commands.
\date{ }



\usepackage{babel}
\usepackage{listings}
\renewcommand{\lstlistingname}{Listing}



\makeatother

\usepackage{babel}
\usepackage{listings}
\renewcommand{\lstlistingname}{Listing}

\begin{document}

\title{Review: Binary Files}
\maketitle

\section*{Overview}

A binary file is a file that is stored in non-text format, instead
of readable characters. Creating and storing binary files are a fundamental
component of computer programing that must be integrated into the
development of your labs and assignments throughout the course. We
will discuss the reasoning and benefits to the topic, including a
``Getting Started'' example, and additional resources.

\section*{Learning Outcomes \label{sec:Learning-Outcomes}}

Upon successful completion of this review, you should be able to:
\begin{itemize}
\item recognize the benefits of binary files
\item create a binary file from a programming example
\item execute a binary
\end{itemize}

\section*{Introduction \label{sec:Introduction}}

A binary file is an non-text file, this means that the content written
within the file is not text characters. Examples of binary files include
but are not limited to image files, compressed files, and computer
programs. For instance an image files often contains brief XML data
about how the image is store followed by a large matrix of numerical
data representing the colors or pixels of an image. Many types of
binary files can be opened with a regular text editor, however the
result is often a file filled with unreadable content.

Binary files provide an more efficient method of storing data files
in comparison to text files. If you consider zipping a regular
text file, the zipped file is a binary file. The zipped file is smaller
size, which requires less memory to store on the system and faster
network transfer speeds.

Software code is stored in a text file, when we compile the software
code with GCC or other compilers, the code is convert to a machine
readable binary format. In computer science, we usually refer to the
binary/hexadecimal formatted file created by the compiler as the program
or executable.

Any computer program needs to be converted to a machine readable format
before execution. This allows the computer to interpret the instructions
in a language it understands. Compiled computer programs are often
stored in a binary file in hexadecimal, also know as hex, which is
a set of hexadecimal numbers representing the original instructions.

\section*{Getting Started \label{sec:Getting-Started}}

Most binary files types have a specific program that opens or executes
the file type. For instance most operating systems contains an image
viewer program that opens the most common types of images.

If you want to look at a compiled binary file for analysis, it may be
helpful to use a program that interprets the binary as hexadecimal.
You can use a hex viewer to interpret the contents of an executable, and
this is often done in the process or reverse engineering. Viewing hexadecimal
of your program or an output file can be very helpful since most debuggers
help translate the errors back into a human readable format.

To generate a binary file of a computer program, we simply need to
compile it.

\begin{lstlisting}[language=make,showstringspaces=false,tabsize=4,frame=TB]
gcc yourFile.c -o myProgram
\end{lstlisting}

The executable myProgram will be a binary file, the contents of which
will be hexadecimal values. As previously specified most binary files
need a specific program to open, however compiled programs can simply
be executed from the command line by calling:

\begin{lstlisting}[language=make,showstringspaces=false,tabsize=4,frame=TB]
./myProgram
\end{lstlisting}


\section*{Examples \label{sec:Examples}}

As previously mentioned, storing files in binary format reduces the
storage requirements of the system and also improves speed of access.
When reading and writing binary files, there is the added advantage
that you can instantly read from any location in the file, more specifically
you can ``seek" to a position to start reading, without having to read any
previous or irrelevant sections. This can not be easily accomplished with
text files.

The following example, binaryExample.c demonstrates the key functions
involved in creating and reading binary files. The following code
is the full program, but we will look at each binary related step
individually. This program attempts to read a binary file with 5 2D
points. If the file does not exist it generates a file with expected
data. Note that data is stored in a binary file, if you open testFile.bin
(bin for binary) in a text editor it will be unreadable.

\begin{lstlisting}[language=C,showstringspaces=false,tabsize=4,frame=TB]
#include<stdio.h>

typedef struct {
        int x,y;
} Point2D;

int main( int argc, char ** argv ) {
    FILE * file = NULL;
    Point2D pt;

    file = fopen("testFile.bin", "rb");

    if( !file ) {

        // If the testFile bin does not already exist, then
        // populate it wth some data.

        file  = fopen("testFile.bin", "wb");

        // Return can't read or write
        if(!file){
            printf("Unable to read or write file.");
            return 1;
        }

        for( int i = 0; i < 5; i++ ) {
            pt.x = i;
            pt.y = i * 3;

            fwrite(&pt,sizeof(Point2D), 1, file);
        }

        printf("Testfile.bin has been written, it's a binary file.\n");
        printf("Open it with a text editor to see it's not readable\n");
        printf("rerun this program to view the data that ");
        printf("was created in binary.\n");
    }
    else {
        // Read all of the examples and print
        for( int i = 0; i < 5; i++ )
        {
            fread(&pt, sizeof(Point2D), 1, file );
            printf("Data was read: %d %d\n", pt.x, pt.y );
        }

        printf("**********************Seek File position\n");

        // Print a specific example, seek a specific position in the file
        // The read from that position
        fseek(file,sizeof(Point2D)*2, SEEK_SET);
        fread(&pt, sizeof(Point2D), 1, file );
        printf("Data was read: %d %d\n", pt.x, pt.y );

        printf("**********************Rewind File\n");


        // Print the first value again, because we've rewound the file.
        rewind(file);
        fread(&pt, sizeof(Point2D), 1, file );
        printf("Data was read: %d %d\n", pt.x, pt.y );
    }
    fclose(file);
    return 0;
}
\end{lstlisting}

The first thing to note about this program is how to open and read
binary files, the process is nearly identical to opening a normal
test file.

\begin{lstlisting}[language=c,showstringspaces=false,tabsize=4,frame=TB]
// Add a b to open a binary file for reading
file = fopen("testFile.bin", "rb");

// Add a b to open a binary file for writing
file = fopen("testFile.bin", "wb");
\end{lstlisting}
After a binary file is open the reading and writing actions are:

\begin{lstlisting}[language=c,showstringspaces=false,tabsize=4,frame=TB]
// Read a Point2D, the size of struct, number to read from which file.
fread(&pt, sizeof(Point2D), 1, file );

// Writing in nearly identical
fwrite(&pt,sizeof(Point2D), 1, file);
\end{lstlisting}

Finally, one of the main benefits of a binary file is that you can
easily control your reading and writing position in the file.

For instance, if you want read a specific example, you simply need
to set the position to start reading from using seek() then read from
that position. The binary data in this file is stored in array of
binary data, so to find a specific position in the file, your program
must indicate the size of the items in the array and the index of
the item in the array. The fseek function takes the file, the position
to start and ENUM indicating the method to apply, in this case we
are setting the position.

\begin{lstlisting}[language=c,showstringspaces=false,tabsize=4,frame=TB]
// Print a specific example, seek a specific position in the file
// The read from that position
fseek(file,sizeof(Point2D)*2, SEEK_SET);
fread(&pt, sizeof(Point2D), 1, file );
printf("Data was read: %d %d\n", pt.x, pt.y );

printf("**********************Rewind File\n");

// Print the first value again, because we've rewound the file.
rewind(file);
fread(&pt, sizeof(Point2D), 1, file );
printf("Data was read: %d %d\n", pt.x, pt.y );
\end{lstlisting}

The rewind function can be used to reset to positional information
of the file, we could easily have used fseek(file,0, SEEK\_SET) too.
Other helpful functions exist for navigation the position of the file
reader, please see additional information for links.

\section*{Additional Information \label{sec:Additional-Information}}
\begin{itemize}
\item \url{https://www.codingunit.com/c-tutorial-binary-file-io} 
\item \url{http://www.cprogramming.com/tutorial/cfileio.html}
\end{itemize}

\end{document}
