%% LyX 2.2.1 created this file.  For more info, see http://www.lyx.org/.
%% Do not edit unless you really know what you are doing.
\documentclass[12pt,english]{article}
\renewcommand{\familydefault}{\sfdefault}
\usepackage{geometry}
\geometry{verbose,tmargin=1in,bmargin=1in,lmargin=1cm,rmargin=1in}
\usepackage{url}
\usepackage{amsmath}
\usepackage{amsthm}

\makeatletter
%%%%%%%%%%%%%%%%%%%%%%%%%%%%%% Textclass specific LaTeX commands.
\numberwithin{equation}{section}
\numberwithin{figure}{section}
\newenvironment{lyxlist}[1]
{\begin{list}{}
{\settowidth{\labelwidth}{#1}
 \setlength{\leftmargin}{\labelwidth}
 \addtolength{\leftmargin}{\labelsep}
 \renewcommand{\makelabel}[1]{##1\hfil}}}
{\end{list}}

%%%%%%%%%%%%%%%%%%%%%%%%%%%%%% User specified LaTeX commands.
\date{ }

\makeatother

\usepackage{babel}
\usepackage{listings}
\renewcommand{\lstlistingname}{Listing}

\begin{document}

\title{Review: GCC}
\maketitle

\section*{Overview}

The GNU Compiler Collection better known as (GCC) is a set of compilers
created by the GNU project. GCC is the main compiler used throughout
the course to turn your source code into an executable. This review
will introduce the usage of GCC and including a ``Getting Started''
example, additional commands and command-line arguments.

\section*{Learning Outcomes \label{sec:Learning-Outcomes}}

Upon successful completion of this review, you should be able to:
\begin{itemize}
\item be aware of the GCC compiler task and main command line options
\item identify potential language syntax errors within their source code
\item utilize the GCC compiler to compile and link their source code into
software
\end{itemize}

\section*{Introduction \label{sec:Introduction}}

GCC is a set of compilers for many languages, originally GCC was originally
made for only the C-programming language. A compiler is another computer
program that converts your source code written in C and converts it
into an executable machine language. When using GCC to create an executable
the process has two stages the first compiles your software code and
the second links all your code into a single executable. The executable
machine language is contained within an object or .obj files produced
by the compiler. The output from the compiler, ie the object files
are combined together into a single executable file by the linker
during the linking phase. Object files are the result of the compiler
and the executable is the result of object files and the linker. The
two stages in the process can be separated if needed or if the object
files have not changed but the majority of the time we will combine
the two phases in a single command.

\section*{Getting Started \label{sec:Getting-Started}}

To get started with GCC lets compile a single file from the command
line, assume we have the file main.c that is basically empty but contains
a print statement to say hello.

\begin{lstlisting}[language=make,showstringspaces=false,tabsize=4,frame=TB]
#include <stdlib.h>
int main(int argc, char** argv)
{
    printf("Hello\n");
	return 0;
}
\end{lstlisting}

To compile and link this file use the following command.

\begin{lstlisting}[language=make,showstringspaces=false,tabsize=4,frame=TB]
# Compile and link code in the same command
gcc main.c -o sayHello
\end{lstlisting}

This command compiles the code and creates an output file called sayHello,
which is the name of your executable code. If you wanted to separate
the compiling and linking code you would use to calls to GCC as such:

\begin{lstlisting}[language=make,showstringspaces=false,tabsize=4,frame=TB]
# The first command compiles, the second links the object files
gcc -c main.c
gcc main.o -o sayHello.
\end{lstlisting}

Separating the compile and linking steps for most of your programming
will not produce any additional benefits, but is something worth knowing.

If you want to review the options GCC provides for yourself here are
some helpful command line calls.

\begin{lstlisting}[language=make,tabsize=4,frame=TB]
# To get additional help from the command line
man gcc

# For all help
gcc --help

# Or more specifc help, specify one of warning, common, target, or params
# as a help option.
gcc --help=warning
\end{lstlisting}


\section*{Examples \label{sec:Examples} }

Let's examine a couple more practical examples, compiling multiple
files and adding additional flags. Case 1 demonstrates how to compile
multiple .c files into a single executable by simply including the
additional filenames.

\begin{lstlisting}[language=make,showstringspaces=false,tabsize=4,frame=TB]
# Case 1
gcc fileOne.c fileTwo.c -o outProgram
\end{lstlisting}

If you don't want to type out all of the files names to compile your
program, you can use the wildcard '{*}' which indicates all files.
If all your source files are properly stored in src directory, you
could compile them with the following:

\begin{lstlisting}[language=make,showstringspaces=false,tabsize=4,frame=TB]
# Case 1
gcc ./src/* -I ./include
\end{lstlisting}

Notice the code ./src/{*} this indicates to include all files in the
src directory, the -I command line argument indicates that the corresponding
include or .h files can be found in the ./include directory. With
proper project setup your code could be compiled and linked in a single
command, however, the standard is to use MakeFiles ( A review on MakeFiles
is also available ) or some form of automated project building to
ease integretations of others trying to use your code and declare
the exact actions the compiler will take

\subsection*{Useful Flags\label{subsec: Useful flags}}

The following is a list of common and helpful command line arguments
or flags to use when running GCC.
\begin{lyxlist}{00.00.0000}
\item [{-I}] indicates a location for header files, you can use it multiple
times if needed.
\item [{-c}] compile only no link
\item [{-o}] output executable name
\item [{-g}] enable debugging with GDB ( Please see GDB review for additional
support)
\item [{-std}] set to one of the C-Language standards here are some common
values to set c89, c99, c11, gnu11, c++11 or c++14
\item [{-Wall}] additional warnings will be check and evaluated
\item [{-Wextra}] extra warnings and errors will be evaluated beyond -Wall
\item [{-pedantic}] use strict standard
\item [{-pedantic-errors}] Similar to pedantic but convert warnings to
errors
\end{lyxlist}

\section*{Additional Information \label{sec:Additional-Information}}

The following links provide valuable additional introductory documentation
and examples.
\begin{itemize}
\item \url{https://gcc.gnu.org/}
\item \url{https://gcc.gnu.org/onlinedocs/gcc/index.html#Top}
\item \url{http://pages.cs.wisc.edu/~beechung/ref/gcc-intro.html}
\end{itemize}

\end{document}
