%% LyX 2.2.1 created this file.  For more info, see http://www.lyx.org/.
%% Do not edit unless you really know what you are doing.
\documentclass[12pt,english]{article}
\renewcommand{\familydefault}{\sfdefault}
\usepackage{geometry}
\geometry{verbose,tmargin=1in,bmargin=1in,lmargin=1cm,rmargin=1in}
\usepackage{float}
\usepackage{amsmath}
\usepackage{amsthm}
\usepackage{graphicx}
\usepackage{textcomp}

\makeatletter
%%%%%%%%%%%%%%%%%%%%%%%%%%%%%% Textclass specific LaTeX commands.
\numberwithin{equation}{section}
\numberwithin{figure}{section}

%%%%%%%%%%%%%%%%%%%%%%%%%%%%%% User specified LaTeX commands.
\date{ }
\newcommand*{\adt}{Linked List}
\setcounter{section}{1}

\makeatother

\usepackage{babel}
\usepackage{listings}
\renewcommand{\lstlistingname}{Listing}

\begin{document}

\title{Review: Doxygen}
\maketitle

\section*{Overview}

This review will cover an introduction to the Doxygen documenting tool.
This review covers the basics of documentating functions definitions and generating
HTML and LaTeX files from function definitions. Doxygen allows for a simple and elegant
solution of creating documentation to allow ease for use of understanding how to use source code.
Doxygen will be used throughout the course to create documentation.
We will review the basic introduction to the
topic, including a ``Getting Started'' example.

\section*{Learning Outcomes}

Upon successful completion of this review, students should be able
to:
\begin{itemize}
\item use the Doxygen tool for documenting source code
\item recognize the syntax of Doxygen's Doxyfile and comments
\item apply Doxygen to generate HTML and LaTeX files that contain documentation of source code
\item illustrate the importance and ease of documenting source code.
\end{itemize}

\section*{Introduction}

In this review, we will cover the basics of:
\begin{itemize}
\item Doxygen
\end{itemize}

\section*{Getting Started}

Please ensure that you're comfortable with material on general programming,
and how to create useful documentation for code.

\section*{Doxygen}

Doxygen is a tool designed for generating documentation from annotated C++ sources,
 but it also supports other popular programming languages such as C, Objective-C,
 C\#, PHP, Java, Python, IDL (Corba, Microsoft, and UNO/OpenOffice flavors), Fortran, VHDL, Tcl, and to some extent D.

To begin using Doxygen, it must first be downloaded or compiled from source. 
Doxygen is supported by Linux x86-64, Windows, and Mac OS X 10.5 and later, though it may work on other machines.
It's website url is www.doxygen.org/, where its binaries are available.
Doxygen may be compiled from source with dependencies 'make' and 'cmake' on Unix-like machines.

Doxygen is installed on the school computers already, but if you wish to use it at home you'll need to install it.

\medskip{}

Once installed, the program may be invoked at command line via 'Doxygen' (case sensitive).
Before we begin documenting source code, we may create a 'Doxyfile' used to modify
our output file in the directory containing our programming project.

You can start with a blank doxyfile by asking doxygen to create the template for you.
\noindent \begin{center}
\begin{lstlisting}[showstringspaces=false,tabsize=4,frame={T|B}]
	cd [codeProject]
	doxygen -g [name]
\end{lstlisting}
\par\end{center}

You can also use a file from a different project and change the source directory and any other info that needs changing.

\medskip{}

If the Doxyfile is not given a name, it defaults to 'Doxyfile', otherwise the file generated 
will be the name that it is given in the [name] parameter. Now that we have this file,
we may use it to parse our documentation. First, we need to create documentation.
This review will demonstrate an example .h header C file and how to properly comment.



\section*{Using Doxygen}

Doxygen works by parsing comments to find specific markup symbols.  The markup is used to tell doxygen what the different parts of the comments are.  It works very much like the javadoc tool for java documentation, but is available for more languages.

Below is an example .h file:


\begin{lstlisting}[showstringspaces=false,tabsize=4,frame={T|B}]

	struct Example{
		int *z;
	}
	int add( int a, int b );
	float div(int a, int b);
	
\end{lstlisting}

\medskip{}

At the moment, this header file has no documentation whatsoever. To another user, or even
yourself at a future date, you may forget how to use these functions, and may not know what is to 
be expected from their input and outputs. Reading source code is time consuming and often confusing.
To fix this problem we must document the code using comments.

Same .h file with two different types of Doxygen comments:

\begin{lstlisting}[showstringspaces=false,tabsize=4,frame={T|B}]
	
	/**
	 *@file example.h
 	 *@author Michael Ellis
 	 *@date April 2017
 	 *@brief Doxygen example fille.
 	 **/

	/**Structure containing a pointer to an integer.*/
	struct Example{
		//This is for documenting struct elements.
		int *z; ///<A pointer to an integer. 
	}

	/**This function takes in two integers, and adds them together. 
	 *For example if we call the function with parameters of 3 and 4
	 *the result is 3+4, which is 7.
	 *@param a An integer that is to be added with another integer.
	 *@param b An integer that is to be added with another integer.
	 *@return a and b added together.
	 *This can be mathematically expressed as a+b
	 **/
	int add( int a, int b );

	//exclamation mark instead of two stars
	/*!This function takes in two integers, and divides the first one by 
	 *the second. The result of the function is a/b.
	 *@param a The numerator.
	 *@param b The denominator.
	 *@pre b must not be zero, or else there will be division by zero.
	 *@return a/b as floating point number.
	 *If b is zero, return error code as 0.
	 **/
	float div(int a, int b);
\end{lstlisting}


Doxygen looks at comment blocks that begin with a slash followed by two asterisks.   It looks for field names that begin with the @ symbol.   Doxygen has a rich set of operations and functionality. For this course we're really just using the basics to generate documentation.

For more on Doxygen comment blocks see the Doxygen tutorial :

www.stack.nl/{\raise.17ex\hbox{$\scriptstyle\sim$}}dimitri/doxygen/manual/docblocks.html

For more on Doxygen commands, such as 'param' see the Doxygen tutorial :

www.stack.nl/~dimitri/doxygen/manual/commands.html


\section*{Generating HTML and LaTeX}

This sections provides an example of how to generate documentation files out of Doxygen comments.
From the previous section, you had created a Doxyfile. We will now use that file in order to 
finish documenting our commented file. We must now edit our Doxyfile using a text editor of choice.
The file has quite a lot of detail, but this review will only provide an overview of the essentials.

First, we will name our resulting documentation by modifying the 'PROJECT\_NAME' variable of our
Doxyfile. You can name it whatever best fits your project.
\begin{lstlisting}[showstringspaces=false,tabsize=4,frame={T|B}]
	PROJECT_NAME           = "My Project"
	
\end{lstlisting}

 Next, we may want to set our
output directory, which is dictated by the variable: 'OUTPUT\_DIRECTORY'. This is generally
named 'docs' to give users a standard name to search for when looking for documentation.

\begin{lstlisting}[showstringspaces=false,tabsize=4,frame={T|B}]
	OUTPUT_DIRECTORY       = docs
	
\end{lstlisting}
Most importantly, we want to give our Doxygen the location of our file to be documented. 
This is found from the 'INPUT' variable, and should be a path to your file, treating the location
of your Doxyfile as the root directory. If your file was in a directory named 'include/' and your header file was named 'example.h'
you would set my 'INPUT' variable as include/example.h.

\begin{lstlisting}[showstringspaces=false,tabsize=4,frame={T|B}]
	INPUT                  = include/example.h
	
\end{lstlisting}

You may look at the other variables in the Doxyfile, as some
of them may be relevant to your documentation.

Once you have your Doxyfile configured. You can now generate HTML and LaTek files
using the command:

\begin{lstlisting}[showstringspaces=false,tabsize=4,frame={T|B}]
	Doxygen [name of Doxyfile]
	
\end{lstlisting}

Review the output of Doxygen to verify that there are no warnings or errors. Any undocumented
elements will be reported by Doxygen.
Finally, review the output files generated in your specified output directory.

Doxygen has many more features not mentioned in this review. If you are interested in learning more about Doxygen,
review its website's documentation:

http://www.stack.nl/~dimitri/doxygen/manual/index.html

\end{document}
