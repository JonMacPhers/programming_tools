%% LyX 2.2.1 created this file.  For more info, see http://www.lyx.org/.
%% Do not edit unless you really know what you are doing.
\documentclass[12pt,english]{article}
\renewcommand{\familydefault}{\sfdefault}
\usepackage{geometry}
\geometry{verbose,tmargin=1in,bmargin=1in,lmargin=1cm,rmargin=1in}
\usepackage{url}
\usepackage{amsmath}
\usepackage{amsthm}

\makeatletter
%%%%%%%%%%%%%%%%%%%%%%%%%%%%%% Textclass specific LaTeX commands.
\numberwithin{equation}{section}
\numberwithin{figure}{section}
\newenvironment{lyxlist}[1]
{\begin{list}{}
{\settowidth{\labelwidth}{#1}
 \setlength{\leftmargin}{\labelwidth}
 \addtolength{\leftmargin}{\labelsep}
 \renewcommand{\makelabel}[1]{##1\hfil}}}
{\end{list}}

%%%%%%%%%%%%%%%%%%%%%%%%%%%%%% User specified LaTeX commands.
\date{ }

\makeatother

\usepackage{babel}
\usepackage{listings}
\renewcommand{\lstlistingname}{Listing}

\begin{document}

\title{Review: Tar and GZIP}
\maketitle

\section*{Overview}

Tar and GZIP are two programs used to create and extract compressed
archives of files. This process is also known as zipping or compressing
files, and is the process of grouping and compressing the size of a set of
files. In order to submit your labs and assignments throughout the
course, you will need to submit compressed archived files. This review
introduces the basic usage concepts of Tar and GZIP tools, including
a ``Getting Started'' example, and some useful common-line arguments.

\section*{Learning Outcomes \label{sec:Learning-Outcomes}}

Upon successful completion of this review, you should be able to:
\begin{itemize}
\item recognize the benefits of using an archiving program
\item archive and unarchive your projects for submission
\item apply command-line arguments to modify archives
\end{itemize}

\section*{Introduction \label{sec:Introduction}}

GZIP (GNU Zip) and Tar (Tape Archiver) are two programs \textit{often} used in
combination to create a compressed archive or also known as zip file.
Compressed archives are a simple method of keeping all your information
together and reducing the file size. Compressing your files minimizes the space
the files use on the system and the time to transfer the files. The
first part of creating a compressed archive is to organize all of
the desired files into a single archive or tar file, this uses the
tar command. The second part is the compression phase, which compresses
all the files within the tar file using gzip or an equivalent program.
Tar archives should use the extension .tar, while GZIP compressed
files will have the extension .gz. If Tar and GZIP have been used
in combination, the compressed archive will often have both extensions
in the form ``.tar.gz'', or ``.tgz". Tar and GZIP are so frequently used
together, that it is easy to forget that are different programs, especially
as many of the command-line arguments provide by tar support functionality
available in GZIP.

\section*{Getting Started \label{sec:Getting-Started}}

To get started, let's create a tar archive from a programming project,
with the following command line call:

\begin{lstlisting}[language=make,tabsize=4,frame=TB]
tar cvf myProject.tar ./myProjectDirectory
\end{lstlisting}

The command-line arguments perform the following actions:
\begin{lyxlist}{00.00.0000}
\item [{c}] create a new .tar archive
\item [{v}] the verbose flag provides additional output from the program
\item [{f}] indicates the tar archive should be named by the following
            string, in this case ``myProject.tar"
\end{lyxlist}
This call will create a tar file named myProject.tar, when the program
runs it will output all files and paths that have been included in
the tar archive. If you wanted to also compress the tar file at the
same time as creating it, you only need to add the z command line
argument and change the output filename to reflect that it is now
compressed. If you want to list or extract the contents of a compressed
gz file, you can use the following commands.

\begin{lstlisting}[language=make,tabsize=4,frame=TB]
# Create a compressed tar
tar cvfz myProject.tar.gz ./myProjectDirectory

# List files in compressed tar
gzip -l myProject.tar.gz

# Extract tar file from compressed file
gzip -d myProject.tar.gz
\end{lstlisting}

Extracting a tar.gz file using gzip will give you the original tar archive,
however, you'll probably want to extract the original contents from
within the tar file. To remove the contents from a tar or tar.gz file
use the 'x' command-line parameter

\begin{lstlisting}[language=make,tabsize=4,frame=TB]
# Extract file contents from compression file
tar xvf myProject.tar.gz

# Extract file from tar
tar xvf myProject.tar
\end{lstlisting}

On most machines, you can extract the files from the compressed archive
all in one go by running the following command
\begin{lstlisting}[language=make,tabsize=4,frame=TB]
tar xvzf myProject.tar.gz
\end{lstlisting}


\subsection*{Arguments \label{subsec:Arguments}}

Tar contains several useful command-line arguments, this list contains
some of the more common command-line arguments:
\begin{lyxlist}{00.00.0000}
\item [{-d}] find the difference between archive to file system
\item [{-r}] append files to archive
\item [{-t}] list the contents of tar
\item [{-u}] update only archive that are new or changed files
\item [{-x}] extract files from an archive
\item [{-f}] give filename to tar or zip (last argument before filename)
\item [{-c}] create an archive
\item [{-z}] compress the tar files using gzip
\item [{-v}] verbose
\item [{\textendash delete}] delete files form archives
\end{lyxlist}

\section*{Additional Information \label{sec:Additional-Information}}
\begin{itemize}
\item \url{http://www.tecmint.com/18-tar-command-examples-in-linux/}
\item \url{https://www.gnu.org/software/tar/}
\item \url{https://www.howtogeek.com/248780/how-to-compress-and-extract-files-using-the-tar-command-on-linux/}
\item \url{http://linuxcommand.org/man_pages/tar1.html}
\end{itemize}

\end{document}
