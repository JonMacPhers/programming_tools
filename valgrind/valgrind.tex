%% LyX 2.2.2 created this file.  For more info, see http://www.lyx.org/.
%% Do not edit unless you really know what you are doing.
\documentclass[12pt,english]{article}
\renewcommand{\familydefault}{\sfdefault}
\usepackage{geometry}
\geometry{verbose,tmargin=1in,bmargin=1in,lmargin=1cm,rmargin=1in}
\usepackage{url}
\usepackage{amsmath}
\usepackage{amsthm}

\makeatletter
%%%%%%%%%%%%%%%%%%%%%%%%%%%%%% Textclass specific LaTeX commands.
\numberwithin{equation}{section}
\numberwithin{figure}{section}

%%%%%%%%%%%%%%%%%%%%%%%%%%%%%% User specified LaTeX commands.
\date{ }

\makeatother

\usepackage{babel}
\usepackage{listings}
\renewcommand{\lstlistingname}{Listing}

\begin{document}

\title{Review: Valgrind}
\maketitle

\section*{Overview}

Valgrind is a suite of tools packaged together to provide additional
debugging, memory checking, and profiling capabilities. This review
will introduce Valgrind and demonstrate how to get started with the
suite of tools for evaluating your source code.

\section*{Learning Outcomes \label{sec:Learning-Outcomes}}

Upon successful completion of this review, you should be able to:
\begin{itemize}
\item different the benefits and capabilities of Valgrind
\item use Valgrind to evaluate your software project
\item maximize testing efficiency by using a range Valgrind tools
\end{itemize}

\section*{Introduction \label{sec:Introduction}}

Valgrind is a suite of tools packaged together to provide improved
run-time debugging and memory checking. Valgrind is best supported
through Linux and Mac OS. Valgrind operates on your executable by
adding additional debugging information. The most popular Valgrind
tool is memcheck which detects memory-management problems, such as
memory leaks or invalid frees. Memcheck is a very useful tool for
identifying potential run time errors related to memory problems.
Other Valgrind tools search for memory usage problems in the cache,
the heap, or threads.

\section*{Getting Started \label{sec:Getting-Started}}

To get started with Valgrind, compile your project and create an executable.
The review material on GCC or Makefiles can help you get started with
compiling your code. Once the executable is compiled and linked, you
can use Valgrind to evaluate your program by running:

\begin{lstlisting}[language=make,showstringspaces=false,tabsize=4,frame=TB]
Valgrind ./yourExecutableName
\end{lstlisting}

By default, Valgrind runs only memcheck, the output from memcheck
can be quite verbose. The output from Valgrind-memcheck contains three
portions, the first is the program output, followed by two summary
sections for the heap and leaks. The summary section can be extended
by rerunning Valgrind with \textendash leak-check-full command line
parameter. By default, Valgrind will indicate how many blocks of memory
where used and freed. If a memory leak does occur and you run Valgrind
with the \textemdash leak-check-full command you will be provided
with additional information on the leak. Additional information can
include the name of the variable, the line number the object was created
on, and the filename. The caveat to using memcheck is that certain
static and global variables might not appear to have proper memory
management because they are not freed until the very end of the program. 

Here is an example of some of the output you might see from memcheck.

\begin{lstlisting}[language=make,basicstyle={\scriptsize},showstringspaces=false,tabsize=4,frame=TB]
==4560== HEAP SUMMARY:
==4560==     in use at exit: 20 bytes in 2 blocks
==4560==   total heap usage: 5 allocs, 3 frees, 1,072 bytes allocated
==4560== 
==4560== LEAK SUMMARY:
==4560==    definitely lost: 16 bytes in 1 blocks
==4560==    indirectly lost: 4 bytes in 1 blocks
==4560==      possibly lost: 0 bytes in 0 blocks
==4560==    still reachable: 0 bytes in 0 blocks
==4560==         suppressed: 0 bytes in 0 blocks
==4560== Rerun with --leak-check=full to see details of leaked memory
==4560== 
==4560== For counts of detected and suppressed errors, rerun with: -v
==4560== ERROR SUMMARY: 0 errors from 0 contexts (suppressed: 0 from 0)
\end{lstlisting}

To change tools from memcheck, use the \textendash tool= command line
parameter. The other tools are named Cachegrind (Cache Profiler),
Callgrind (Cache Extension), Massif (Heap Profiler), Helgrind (Thread
Debugger ), DRD ( Multithreaded errors).

Here is an example of using the Valgrind with Cachegrind tool.

\begin{lstlisting}[language=make,showstringspaces=false,tabsize=4,frame=TB]
Valgrind --tool=cachegrind ./yourExecutableName
\end{lstlisting}


\section*{Additional Information \label{sec:Additional-Information}}
\begin{itemize}
\item \url{http://Valgrind.org/docs/manual/quick-start.html}
\item \url{http://Valgrind.org/info/tools.html}
\item \url{http://cs.ecs.baylor.edu/~donahoo/tools/Valgrind/}
\end{itemize}

\end{document}
