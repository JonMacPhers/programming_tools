%% LyX 2.2.2 created this file.  For more info, see http://www.lyx.org/.
%% Do not edit unless you really know what you are doing.
\documentclass[12pt,english]{article}
\renewcommand{\familydefault}{\sfdefault}
\usepackage{geometry}
\geometry{verbose,tmargin=1in,bmargin=1in,lmargin=1cm,rmargin=1in}
\usepackage{url}
\usepackage{amsmath}
\usepackage{amsthm}

\makeatletter
%%%%%%%%%%%%%%%%%%%%%%%%%%%%%% Textclass specific LaTeX commands.
\numberwithin{equation}{section}
\numberwithin{figure}{section}

%%%%%%%%%%%%%%%%%%%%%%%%%%%%%% User specified LaTeX commands.
\date{ }

\makeatother

\usepackage{babel}
\begin{document}

\title{Review: Virtualbox}
\maketitle

\section*{Overview}

Virtualbox is a useful tool for emulating different operating systems
or specific hardware. When developing software for a specific system,
if you are unable to use the exact software or exact hardware, the
next best thing is to use an emulator to mimic the behavior of the
physical environment. This review will introduce Virtualbox and some
helpful resources for getting started.

We have a virtualbox-ready image of the SOCS linux system that you can download and install on your own computer if you wish to have a linux system to work on without connecting via ssh or nomachine.   Check the course website for information.

\section*{Learning Outcomes \label{sec:Learning-Outcomes}}

Upon successful completion of this review, you should be able to:
\begin{itemize}
\item recognize the benefits of virtual box
\item be aware of helpful resources for getting setup
\end{itemize}

\section*{Introduction \label{sec:Introduction}}

Virtualbox is a virtualization system which allows users to run one
or more operating systems on the same machine at the same time. The
virtualized system mimics the behavior of an actual physical system.
Creating an environment developers can use to build and test their
software without having access to the real requirements. Virtualization
systems do their best to replicate the actual environment but there
will be differences between the virtual system and real system. Virtualization
allows for installation to be standardized and reduces the cost of
requiring a physical machine for each testing environment.

The original operating system and physical hardware are referred to
as the host system, any virtual environments that are setup are called
clients. The host environment is protected from the actions that occur
on client systems via a process called sandboxing. Sandboxing means
that the software has restricted access to only a small amount of
memory space preallocated on the host system.

The Virtualbox website is located at \url{https://www.virtualbox.org/}and
is available for all major operating systems, to get started download
the VirtualBox need for your host system at \url{https://www.virtualbox.org/wiki/Downloads}.

\section*{Getting Started \label{sec:Getting-Started}}

The process of getting setup on VirtualBox is beyond the scope of
this review and the official getting started guide is update-to-date
and maintained (See Section 1.6 of \url{https://www.virtualbox.org/manual/ch01.html}
). Virtualization systems use the resources of the host thus when
getting started you will want to select options that ensure both the
host and client have enough resources to operate. 

During VirtualBox setup, a user can specify different operating system
features to emulate, such as the size of the hard-drive or the amount
of RAM. If you which to use a premade operating system, many virtual
OS systems are setup at \url{http://www.osboxes.org/}.

Once you've installed the basic Virtualbox certain features will be
useful such as installing guest additions which allows many features
to interact between client and host setup ( \url{https://www.virtualbox.org/manual/ch04.html}
)

\section*{Additional Information \label{sec:Additional-Information}}
\begin{itemize}
\item \url{https://www.virtualbox.org/manual/ch01.html#idm26}
\item \url{http://www.osboxes.org/}
\item \url{http://www.wikihow.com/Install-VirtualBox}
\end{itemize}

\end{document}
