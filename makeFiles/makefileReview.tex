%% LyX 2.2.1 created this file.  For more info, see http://www.lyx.org/.
%% Do not edit unless you really know what you are doing.
\documentclass[12pt,english]{article}
\renewcommand{\familydefault}{\sfdefault}
\usepackage{geometry}
\geometry{verbose,tmargin=1in,bmargin=1in,lmargin=1cm,rmargin=1in}
\usepackage{amsmath}
\usepackage{amsthm}
\usepackage{url}

\makeatletter
%%%%%%%%%%%%%%%%%%%%%%%%%%%%%% Textclass specific LaTeX commands.
\numberwithin{equation}{section}
\numberwithin{figure}{section}

%%%%%%%%%%%%%%%%%%%%%%%%%%%%%% User specified LaTeX commands.
\date{ }

\makeatother

\usepackage{babel}
\usepackage{listings}
\renewcommand{\lstlistingname}{Listing}

\begin{document}

\title{Review: Make Files}
\maketitle

\section*{Overview}

A Makefile is a text file containing a simple set of instructions
to run your program, it indicates which files should be compiled and
defines common commands. Makefiles are a fundamental component that
must be integrated into the development of your labs and assignments
throughout the course. We will review basic introduction to the topic,
including a ``Getting Started'' example, and some common best practice 

\section*{Learning Outcomes \label{sec:Learning-Outcomes}}

Upon successful completion of this review, you should be able to:
\begin{itemize}
\item recognise the benefits of make and makefile 
\item create makefile variables and rules
\item develop your own makefile for labs and assignments
\end{itemize}

\section*{Introduction \label{sec:Introduction}}

A makefile is a simple text file, that is built in any text editor,
it contains rules for performing tasks or running commands, such as
compiling and linking your program. The makefile is used by the program
make ( https://www.gnu.org/software/make/) to customise the behaviour
of the compilation tools and the output produced. The makefile contains
rules and instructions for building your specific project, it contains
similar commands to those manually typed into the command line to
compile your project. The main benefits of using a makefile are that
it reduces the number of commands you have to continuously write to
recompile or test your code, it also only recompiles code that has
changed and it can or partially be reused between labs and projects.

A makefile consists of a number of rules, each rule is defined in
the using a specific format.
\begin{center}
\begin{lstlisting}[language=make,showstringspaces=false,tabsize=4,frame=TB]
name: dependencies
	system command(s)
\end{lstlisting}
\par\end{center}

Each rule will follow this format, it essentially has three parts.
The first part is the rule name which can be anything you want. The
second part is the rule's dependencies this indicates what files and
references your rule will need to complete. For example, perhaps you
submitted a .c and .h file these will depend on each other. The final
part of a rule is the system commands which run a command with the
dependencies, this will be empty or single or set of system commands,
system commands can be tested in your command line before putting
them into a makefile. Notice that the system commands are indented
with a tab, in makefiles the differences between tabs and spaces is
important. You must use tabs instead of spaces for indentation of
any commands in your make files, otherwise, you will experience compile
errors on the otherwise functional code.

\section*{Getting Started \label{sec:Getting-Started}}

The make program should be installed on all university computer science
machines, if you type make without a makefile into the command line
you should receive an error indicating '{*}{*}{*} no targets specified
.... '. If you name your makefile 'makefile', the make program will
automatically use it, if your makefile has a unique name, you will
have to use the command ``make -f yourMakeFileName''. Let's develop
a simple example to demonstrate rule creation, the following code
would be in the makefile.

\begin{lstlisting}[language=make,tabsize=4,frame=TB]
sayHello: 
	@echo 'hello'

clear:
	clear; ls
\end{lstlisting}

In this example, we have created two rules one called 'sayHello' and
another called 'clear'. Both rules call a system command, the 'sayHello'
rule calls echo and 'clear' rule calls clear then calls list. Our
first rules 'sayHello', is the default rule that will be run automatically
if you type make in the command line. The response should be 'hello'
written back in the console. Notice, we have to put @ symbol in front
of echo command, this suppresses other output by make, and simply
echo's the statement 'hello'. For more information on the echo command,
please see ( https://www.gnu.org/software/make/manual/html\_node/Echoing.html).
Our second rule 'clear', first clears the screen and then runs ls
command to list files in the current directory. You can run a specific
rule by typing: ``make yourRuleName'', in this case, run: ``make
clear'' without quotes. The result of running this rule is the screen
is cleared and your files should be listed on the top line. If you
have multiple commands you which to run within the same rule you can
separate commands by using ';'.

\subsection*{Variables \label{subsec:Variables}}

In order to create your own variables in a makefile simply type, variableName
= someValue. To use the value of your variable in the makefile simply
enclose it in \$(yourVariable) such as \$(CC). Here are some common
variables you'll see in this course:

\begin{lstlisting}[language=make,showstringspaces=false,tabsize=4,frame=TB]
# This is a comment in a makefile.
CC = gcc
CFLAGS = -Wall
OBJDIR = obj/

# To display the value of a variable you can use.
@echo $(CC)
\end{lstlisting}


\section*{Examples \label{sec:Examples} }

Let's examine a more practical example, here is an example of a makefile
that contains code to build a program for array example and the array
test cases.

\begin{lstlisting}[language=make,showstringspaces=false,tabsize=4,frame=TB]
# Flags that are sent to the compiler
CC          =   gcc 
CFLAGS      =   -Wall -std=c99 -pedantic -g

#Directories where the compiler can find things
INCLUDES    = -Iinclude

# add directory names here if you want to separate files by directories
BINDIR = bin/
SRCDIR = src/
OBJDIR = obj/

#Put the names of your source code file in the lines below.
SOURCE = $(SRCDIR)test.c $(SRCDIR)array.c $(SRCDIR)testMain.c 
OBJS = $(OBJDIR)test.o $(OBJDIR)array.o

#The names of the binary programs that will be produced. 
PROGNAME = $(BINDIR)arrayExample.out
PROGTEST = $(BINDIR)arrayTest.out

# This target (command) handles the basic make, creates a program with program Name 
# Note this  command doesn't actually build all, it build the main program only, ie not the tests. 

all : program test

program: $(OBJS) $(OBJDIR)main.o  
      $(CC) $(OBJS) $(OBJDIR)main.o -o $(PROGNAME)

# This target (command) handles make test 
test: $(OBJS) $(OBJDIR)testMain.o  
      $(CC) $(OBJS) $(OBJDIR)testMain.o -o $(PROGTEST)

# Below here are targets for *.o
$(OBJDIR)array.o: $(SRCDIR)array.c   
     $(CC) $(CFLAGS) $(INCLUDES) -c $(SRCDIR)array.c -o $(OBJDIR)array.o

$(OBJDIR)test.o: $(SRCDIR)test.c     
	  $(CC) $(CFLAGS) $(INCLUDES) -c $(SRCDIR)test.c -o $(OBJDIR)test.o

$(OBJDIR)testMain.o: $(SRCDIR)testMain.c  
      $(CC) $(CFLAGS) $(INCLUDES) -c $(SRCDIR)testMain.c -o $(OBJDIR)testMain.o

$(OBJDIR)main.o: $(SRCDIR)main.c
	  $(CC) $(CFLAGS) $(INCLUDES) -c $(SRCDIR)main.c -o $(OBJDIR)main.o

# This is another make command, run: make clean and it removes *.o and programs.
# You can make similar commands to perform different common actions. 
clean:   @ rm $(OBJDIR)*.o  $(BINDIR)$(PROGNAME) $(BINDIR)$(PROGTEST) 
\end{lstlisting}


\subsection*{Example Walk through \label{subsec:Example-Walk-through}}

The first section (shown again below) sets a number of variables about
our project, a variable (CFLAGS) for the flags we want to pass to
our compiler, INCLUDES variable for the compiler to know the directory
where to look for include files ( .h ). SOURCE and OBJS variables
set to store our .c files and object files respectively. Note that
we will create the individual rules for generating our object files
further down in the makefile.

\begin{lstlisting}[language=make,showstringspaces=false,tabsize=4,frame=TB]
# Flags that are sent to the compiler
CC          =   gcc 
CFLAGS      =   -Wall -std=c99 -pedantic -g

#Directories where the compiler can find things
INCLUDES    = -Iinclude

# add directory names here if you want to separate files by directories
BINDIR = bin/
SRCDIR = src/
OBJDIR = obj/

#Put the names of your source code file in the lines below.
SOURCE = $(SRCDIR)test.c $(SRCDIR)array.c $(SRCDIR)testMain.c 
OBJS = $(OBJDIR)test.o $(OBJDIR)array.o

#The names of the binary programs that will be produced. 
PROGNAME = $(BINDIR)arrayExample.out
PROGTEST = $(BINDIR)arrayTest.out
\end{lstlisting}

The second section of the makefile creates three rules. The first
rule called 'all' is the default rule that simply tells make to run
both program and test rules. The second rule called 'program' compiles
our program, it uses the object files as the dependencies and the
system call is the GCC call to compile with our object Files and rename
the executable to our program name. The proper object files have not
been created at this point, thus the program must use some of the
.o rules to create the appropriate .o files. The third command called
'test', does exactly the same thing as 'program' command but using
a different main to run the tests instead of the program and renames
the executable to arrayTest.out.

\begin{lstlisting}[language=make,showstringspaces=false,tabsize=4,frame=TB]
all : program test

program: $(OBJS) $(OBJDIR)main.o  
      $(CC) $(OBJS) $(OBJDIR)main.o -o $(PROGNAME)

# This target (command) handles make test 
test: $(OBJS) $(OBJDIR)testMain.o  
      $(CC) $(OBJS) $(OBJDIR)testMain.o -o $(PROGTEST)
\end{lstlisting}

Next, our makefile needs rules to develop all of the object files
to properly compile this program. Below is the original code to compile
all of the object files, each rule uses the compiler and compiler
flags to combine the source file (.c) and uses the Includes variable
to include the directory for the proper header file. Most of the variables
in this example are pointing to the position of the files in their
directories.

\begin{lstlisting}[language=make]
# Below here are targets for *.o
$(OBJDIR)array.o: $(SRCDIR)array.c   
     $(CC) $(CFLAGS) $(INCLUDES) -c $(SRCDIR)array.c -o $(OBJDIR)array.o

$(OBJDIR)test.o: $(SRCDIR)test.c     
	  $(CC) $(CFLAGS) $(INCLUDES) -c $(SRCDIR)test.c -o $(OBJDIR)test.o

$(OBJDIR)testMain.o: $(SRCDIR)testMain.c  
      $(CC) $(CFLAGS) $(INCLUDES) -c $(SRCDIR)testMain.c -o $(OBJDIR)testMain.o

$(OBJDIR)main.o: $(SRCDIR)main.c
	  $(CC) $(CFLAGS) $(INCLUDES) -c $(SRCDIR)main.c -o $(OBJDIR)main.o
\end{lstlisting}

At this point, the makefile contains all rules required to properly
build your project, the test program, or any of specific object file.
The last section contains helper functions to be used during development.
Finally, we make one rule to clean our project, you should clean your
project before zipping for submission or if one of the object files
appears to be stale and is not being updated properly.

\begin{lstlisting}[language=make,showstringspaces=false,tabsize=4,frame=TB]
# This is another make command, run: make clean and it removes *.o and programs.
# You can make similar commands to perform different common actions. 
clean:   @ rm $(OBJDIR)*.o  $(BINDIR)$(PROGNAME) $(BINDIR)$(PROGTEST) 
\end{lstlisting}

To use the makefile 

\begin{lstlisting}[language=make,showstringspaces=false,tabsize=4,frame=TB]
#Build everything your project
make

# Build only the test project
make test

# Rebuild your array object file
make /obj/array.o

# Clean your project for submission
make clean
\end{lstlisting}


\section*{Additional Information \label{sec:Additional-Information}}

The creation of rules for each header and source file can become tedious
with larger projects, for additional resources on how to use built
in macros to build rules, check out this good reference \url{https://www.cs.rochester.edu/\textasciitilde{}nelson/courses/csc\_173/review/make.html}.
\end{document}
