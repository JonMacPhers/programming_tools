%% LyX 2.2.2 created this file.  For more info, see http://www.lyx.org/.
%% Do not edit unless you really know what you are doing.
\documentclass[12pt,english]{article}
\renewcommand{\familydefault}{\sfdefault}
\usepackage{geometry}
\geometry{verbose,tmargin=1in,bmargin=1in,lmargin=1cm,rmargin=1in}
\usepackage{url}
\usepackage{amsmath}
\usepackage{amsthm}

\makeatletter
%%%%%%%%%%%%%%%%%%%%%%%%%%%%%% Textclass specific LaTeX commands.
\numberwithin{equation}{section}
\numberwithin{figure}{section}

%%%%%%%%%%%%%%%%%%%%%%%%%%%%%% User specified LaTeX commands.
\date{ }

\makeatother

\usepackage{babel}
\usepackage{listings}
\renewcommand{\lstlistingname}{Listing}

\begin{document}

\title{Review: SFTP}
\maketitle

\section*{Overview}

SFTP stands for SSH or Secure File Transfer Protocol, it is used to
securely transfer files between remote servers. It is useful for transfer
files between your personal computer and the university development
servers. We will review a basic introduction to the topic, including
a ``Getting Started'' example, and transferring files. 

\section*{Learning Outcomes \label{sec:Learning-Outcomes}}

Upon successful completion of this review, you should be able to:
\begin{itemize}
\item recognize the benefits of SFTP
\item connecting to remote servers
\item transferring files between remote machines
\end{itemize}

\section*{Introduction \label{sec:Introduction}}

FTP is a very common network protocol used throughout the Internet
similar to protocols such as TCP/IP or HTTP. It is used to securely
transfer files between a client and server. SFTP is similar to SSH
in that it provides a secure session for user interaction.Review material
for SSH is available in the course documents which demonstrates securely
connecting to another remote machine. FTP can be used through an SSH
session but the SSH produce is also used to authenticate the SFTP
protocol.

Transferring files can occur using the SFTP command or putty using
the command line or more commonly through an application interface
such as FileZilla, WinSCP, or FireFTP.

\section*{Getting Started \label{sec:Getting-Started}}

SFTP is a command you can run from the terminal: 
\begin{lstlisting}[language=make,showstringspaces=false,tabsize=4,frame=TB]
sftp yourUserName@general.uoguelph.ca
\end{lstlisting}
. If you have trouble connecting you can try connecting with SSH first.
Once you have used the SFTP command and connected using your password.
You can identify the list of available commands by typing '?'. To
transfer files use the command
\begin{lstlisting}[language=make,showstringspaces=false,tabsize=4,frame=TB]
get remoteFile localFile
\end{lstlisting}
. Alternatively, if you want to put a file on a remote host, use the
following command:

\begin{lstlisting}[language=make,showstringspaces=false,tabsize=4,frame=TB]
put yourUserName@general.uoguelph.ca
\end{lstlisting}
.

Filezilla is an application that manages the SFTP connection and provides
an easy to use GUI for transferring files between client and the survey.
To get started download the FileZilla client application. In the Filezilla
network options, you will need to provide the same network information
used to connect to the linux.socs.uoguelph.ca site, this includes
your username, the server (linux.socs.uoguelph.ca), and your password.
A small tutorial for the Filezilla user interface is available at
\url{https://wiki.filezilla-project.org/FileZilla_Client_Tutorial_(en)}.
Once you've established a connection with Filezilla, transferring
files will be a simple drag-and-drop interface between the client
and the server.

\section*{Additional Information \label{sec:Additional-Information}}
\begin{itemize}
\item \url{https://filezilla-project.org/}
\item \url{https://www.digitalocean.com/community/tutorials/how-to-use-sftp-to-securely-transfer-files-with-a-remote-server}
\end{itemize}

\end{document}
