%% LyX 2.2.1 created this file.  For more info, see http://www.lyx.org/.
%% Do not edit unless you really know what you are doing.
\documentclass[12pt,english]{article}
\renewcommand{\familydefault}{\sfdefault}
%\usepackage[T1]{fontenc}
\usepackage[latin9]{inputenc}
\usepackage{geometry}
\geometry{verbose,tmargin=1in,bmargin=1in,lmargin=1cm,rmargin=1in}
\usepackage{amsmath}
\usepackage{amsthm}

\makeatletter
%%%%%%%%%%%%%%%%%%%%%%%%%%%%%% Textclass specific LaTeX commands.
\numberwithin{equation}{section}
\numberwithin{figure}{section}

%%%%%%%%%%%%%%%%%%%%%%%%%%%%%% User specified LaTeX commands.
\date{ }
\newcommand*{\adt}{Linked List}
\newcommand*{\submissiontype}{-SET VALUE HERE-}
\newcommand*{\checkhref} { *** }

\makeatother

\usepackage{babel}
\usepackage{listings}
\renewcommand{\lstlistingname}{Listing}

\begin{document}

\title{Review: Void Pointers}
\maketitle

\section*{Overview}

This review will cover an introductory to pointers . This review cover
the basic introductory material, that was provided in other required
courses. Pointers are a fundamental component that must be integrated
into the development of your labs and assignments throughout the course.
We will review basic introduction to the topic, including a ``Getting
Started'' example, and some common best practices.

\section*{Learning Outcomes}

Upon successful completion of this review, students should be able
to:
\begin{itemize}
\item describe the key benefits of pointers
\item explain the difference between referencing and dereferencing pointers
\item apply pointers to the development of an abstract data type
\item manage pointers to reduce memory leaks and errors
\end{itemize}

\section*{Introduction}

In this review we will cover the basics of :
\begin{itemize}
\item pointers
\item void pointers
\end{itemize}

\section*{Getting Started}

\subsection*{Declaring and Initializing Pointers}

A pointer variable is a variable that points to the memory address
of a variable. That is a pointer is similar to the idea of a container
that contains another variable. For a full reference to pointers review,
see (\checkhref C01-f16-notes Wang). Pointers are declared similar
to other variables except it now includes the {*} operator, this signals
that this variable is a pointer. To declare a pointer to an integer
or a character pointer use the following formats:

\rule[0.5ex]{1\columnwidth}{1pt}

\begin{lstlisting}[tabsize=4]
	int* ptr;
	char* chPtr;
\end{lstlisting}

\rule[0.5ex]{1\columnwidth}{1pt}

\medskip{}

It is generally a good idea to give your pointer a default value or
initialize the pointer if you have no value to set your pointer to
when declared point it to a null. Null is a standard value which can
be evaluated in expressions such as if-statements to indicate no value.

\medskip{}

\rule[0.5ex]{1\columnwidth}{1pt}

\begin{lstlisting}[tabsize=4]
	int* ptr = null;
	char* chPtr = null;
\end{lstlisting}

\rule[0.5ex]{1\columnwidth}{1pt}

\subsection*{Assigning Pointers}

When working with pointers, you have two operators to work with get
the different information about the pointer. The \& symbol refers
to the memory address (also known as the reference operator), the
{*} symbol references the value at that address (known as the dereference
operator). Most often when you're pointing to a variable you'll use
the \& symbol to reference the address and when you're working with
the data you'll use the {*} to dereference the memory address for
a value.

\medskip{}

\rule[0.5ex]{1\columnwidth}{1pt}

\begin{lstlisting}[tabsize=4]
	int a = 5;
	
	// We point to the memory space
	int *aptr = &a;
	
	// we want the value at memory space 'a', ie the number 5
	int b = *a;
\end{lstlisting}

\rule[0.5ex]{1\columnwidth}{1pt}

\subsection*{Static vs Dynamic Memory}

Pointers can point to static or dynamic memory, for a full reference
of static and dynamic memory please see (\checkhref C01-f16-notes
Wang). Dynamic memory is any memory space you specifically allocate
with a malloc command, if you allocate memory you must also free it
after your program has finished using it.

\medskip{}

\rule[0.5ex]{1\columnwidth}{1pt}

\begin{lstlisting}[tabsize=4]
	//Examples of static memory
	int n;
	char str[6];
	
	// Examples of dynamic memory with 100-character string called description
	char *description = null;
	description = malloc( 100 * sizeof(char)
	....
	
	// Later we must release this allocated memory.
	free(description)
\end{lstlisting}

\rule[0.5ex]{1\columnwidth}{1pt}

\subsection*{Void Pointers}

A void pointer is similar to a regular pointer, except that the type
is not declared, it could be considered a generic pointer. Void pointers
are useful for algorithms and data structures that perform operations
independent of the type of data. For instance, a sorting algorithm
performs exactly the same independent of the data, as long as the
sorting algorithm knows how to compare two items. To declare a void
pointer, follow the same format as any other pointer.

\medskip{}

\rule[0.5ex]{1\columnwidth}{1pt}

\begin{lstlisting}[tabsize=4]
	void * voidPtr;
\end{lstlisting}

\rule[0.5ex]{1\columnwidth}{1pt}

\medskip{}

Consider that a void pointer hasn't declared a type and thus it can
be used to a point to any variable. Pointers can be assigned to static
variables or can allocate new memory, in these examples we point to
static variables.

\rule[0.5ex]{1\columnwidth}{1pt}

\begin{lstlisting}[tabsize=4]
	void * voidPtr;
	int a = 5;
	float b = 3.57;
	
	/*
	 Assign void pointer to the address of a
	 meaning voidPtr points the memory space for variable a
	*/
	voidPtr = &a;
	
	// Later we can reassign voidPtr to b;
	voidPtr = &b;
\end{lstlisting}

\rule[0.5ex]{1\columnwidth}{1pt}

\subsubsection*{Casting Void Pointers}

The void pointer still remains 'typeless', ie void, but it now points
to a memory location or variable with a type. If we want to use the
information or data in our void pointer, it needs to successfully
cast to a type. For instance, if we want to print a void pointer,
you'll have to cast it to the type contained within the void pointer.

\medskip{}

\rule[0.5ex]{1\columnwidth}{1pt}

\begin{lstlisting}[tabsize=4]
	void * voidPtr;
	int a = 5;
	float b = 3.57;
	
	// Assign void pointer to the address of a
	// meaning voidPtr points the memory space for variable a
	voidPtr = &a;
	print("a void pointer to an integer %d", *( (int*) voidPtr) );
	
	// Later we can reassign voidPtr to b;
	voidPtr = &b;
	
	// To print our float
	print("A void pointer to an integer %f", *( (float*) voidPtr) );
	
\end{lstlisting}

\rule[0.5ex]{1\columnwidth}{1pt}

\medskip{}

The main part of this example is casting the voidPtr to another type,
let's consider what is happening in the cast. Recall to cast a variable
to another variable, we wrap the new type in (newVariableType) .From
the previous example consider this code: {*}( (int{*}) voidPtr), it
contains two parts, the middle (int{*}) voidPtr and the outer dereference:
{*}( the first part). A quick reminder in casting and pointers is
shown below:

\rule[0.5ex]{1\columnwidth}{1pt}

\begin{lstlisting}[tabsize=4]
	int i = 65;
	char ch = (char)i;
	
	// (char*) casts to a character pointer
	int *iptr = &i;
	char *chPtr = (char*)iptr;
\end{lstlisting}

\rule[0.5ex]{1\columnwidth}{1pt}

\medskip{}

This first statement (int{*}) voidPtr casts to an integer pointer
from our original void pointer. The second part of the original statement
is {*}( the first part) dereferences the now integer pointer to get
the actual value store in that memory location of our void pointer.

\medskip{}

\rule[0.5ex]{1\columnwidth}{1pt}

\begin{lstlisting}[tabsize=4]
	// In multiple setups this code: *( (int*) voidPtr)
	// is equivalent to
	
	int * intPtrFromVoid = (int*)voidPtr;
	print("A void pointer to an integer %d", *intPtrFromVoid );
	
\end{lstlisting}

\rule[0.5ex]{1\columnwidth}{1pt}

\subsubsection*{When to Cast Data}

As mentioned void pointers work well when you don't need to know the
type of the data, such as inserting data into a data structure. When
you need to know the data type of the void, for instance, to print
the value, you'll have to cast it to the appropriate data type.

\medskip{}

\rule[0.5ex]{1\columnwidth}{1pt}

\begin{lstlisting}[tabsize=4]
	// Keeping your functions generic with void pointers.
	void insertData( Container* c, void * voidPtr){
		c->insert(voidPtr);
	}
	
	// Cast when you need to know the type.
	void printContainerItem( void * containerItem ){
		int* intPtr = (int*) containerItem;
		print("%d", *intPtr);
		...
	}
\end{lstlisting}

\rule[0.5ex]{1\columnwidth}{1pt}

\section*{Best Practices}

Pointers and void pointers can take a bit of practice to become confident
using in your daily applications. A suggestion when working with pointers
is to remove some of the ugly and difficult to read syntax that inheritance
comes along with pointers. One suggestion is to improve the readability
of your code by using typedefs.

\rule[0.5ex]{1\columnwidth}{1pt}

\begin{lstlisting}[tabsize=4]
	/* 
	Pointers are more difficult to look at .... lets change that.
	Typedefs take the original name and making an alias for it.
	The form is: typedef oldName newName
	To improve the readability of void pointers, we can use
	*/
	typedef void* VoidPtr;
	
	/*
	Now this statement looks more like the variables
	we're used to....although you still have to manage the memory.
	*/
	VoidPtr myData;	
\end{lstlisting}

\end{document}
