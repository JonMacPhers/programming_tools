%% LyX 2.2.2 created this file.  For more info, see http://www.lyx.org/.
%% Do not edit unless you really know what you are doing.
\documentclass[12pt,english]{article}
\renewcommand{\familydefault}{\sfdefault}
\usepackage[latin9]{inputenc}
\usepackage{geometry}
\geometry{verbose,tmargin=1in,bmargin=1in,lmargin=1cm,rmargin=1in}
\usepackage{url}
\usepackage{amsmath}

\makeatletter
%%%%%%%%%%%%%%%%%%%%%%%%%%%%%% User specified LaTeX commands.
\date{ }



\usepackage{babel}
\usepackage{listings}
\renewcommand{\lstlistingname}{Listing}





\usepackage{babel}
\usepackage{listings}
\renewcommand{\lstlistingname}{Listing}

\makeatother

\usepackage{babel}
\usepackage{listings}
\renewcommand{\lstlistingname}{Listing}

\begin{document}

\title{Review: Command Shell}
\maketitle

\section*{Overview}

The command shell, command line, or shell is an user interface which
interacts directly with the operating system. Mastering fundamentals
of the command shell will be integral in developing your labs and
assignments throughout the course. We will discuss the most common
commands, identify the most common types of command shells, discuss
how to call system commands in a ``Getting Started'' example, and
additional resources.

\section*{Learning Outcomes \label{sec:Learning-Outcomes}}

Upon successful completion of this review, you should be able to: 
\begin{itemize}
\item comprehend the difference between CLI and GUI
\item be aware of alternative types of command shell
\item identify the 10 most common commands in bash
\item call shell commands from your C program
\end{itemize}

\section*{Introduction \label{sec:Introduction}}

The command shell is the original user interface for most operating
systems, it allows users to type commands to start programs. While,
many of us have grown accustom to graphical user interface the command-line
interface (CLI) is still a very powerful method of interacting with
the operating system. The command shell is a program that takes user
input, finds then runs the appropriate program with the arguments
provided. On most Linux operating systems, the default shell is bash.
If you a start a terminal in Linux, you'll be using bash. Different
shells have a number of different preferences and commands. Examples
of different shells are zsh, ksh, and tcsh. In addition, there are
multiple terminal applications that can improve usability such xterm,
terminator, and terminal.

As bash is the default command shell, that is where we will focus
our attention. The importance of working from the shell is understanding
how the filesystem is setup. In our GUI, files and folders have a
visual representation that mimicks the underinglying filesystem. Instead
of clicking on folder to enter the folder, you must use the command
(cd) to change directories.

\section*{Getting Started \label{sec:Getting-Started}}

The bash command line provides a powerful method of running software,
one of the key advantages is being able to generate scripts that automate
common commands process. Another advantage of command line software
is being able to quickly share data between programs via a process
called 'piping'.

\subsubsection*{Getting Help}

Most bash commands will provide a helpful description of the program,
the list of possible arguments, and examples. If a bash command or
program has additional information it will usually be available via
one of these standard methods.

\begin{lstlisting}[language=make,showstringspaces=false,tabsize=4,frame=TB]
// provide manual for ls command
man ls

// -h command line argument for help
someCommand -h
\end{lstlisting}


\subsubsection*{Navigation}

Navigation has several shortcuts to refer to specific locations. A
'.' refers to the current directory, while '..' refers to the parent
directory. The '\textasciitilde{}' symbol refers to your home directory.
The '/' symbol refers to the root directory of the computer. To begin
navigating your bash shell several commands are useful

\subsubsection*{List}

To list the contents of a directory, call the command 'ls' followed
by a path to a directory

\begin{lstlisting}[language=make,showstringspaces=false,tabsize=4,frame=TB]
// To list what is the current directory
ls

// List some sub-directory
ls subDirectory/AnotherDir/

// List all hidden files and information
ls -al
\end{lstlisting}


\subsubsection*{Change Directory}

\begin{lstlisting}[language=make,showstringspaces=false,tabsize=4,frame=TB]
// Change directory from your current location
cd subDirectory/anotherSubDirectory
or
cd ./subDirectory/anotherSubDirectory

// Change directory from your home directory.
cd ~/someDirectory/

// Change directory from root directory
cd /var/user/yourUserName
\end{lstlisting}


\subsubsection*{Print Working Directory}

If you would like to view your current directory use the 'pwd' command

\begin{lstlisting}[language=make,showstringspaces=false,tabsize=4,frame=TB]
pwd
\end{lstlisting}


\subsubsection*{Clear screen}

ITo clear the screen of any output use the 'clear' command

\begin{lstlisting}[language=make,showstringspaces=false,tabsize=4,frame=TB]
clear
\end{lstlisting}


\subsubsection*{History}

To view the history of commands called from your command shell use
the 'history' command.

\begin{lstlisting}[language=make,showstringspaces=false,tabsize=4,frame=TB]
history
\end{lstlisting}


\subsubsection*{Previewing Files}

To view the contents of a file there are several functions which are
helpful. Alternative, text-editors such as vim or emacs exist if you
which to edit the contents. To view just the begin of a file use the
'head' command 

\begin{lstlisting}[language=make,showstringspaces=false,tabsize=4,frame=TB]
// View top contents of a file
head someFile.txt

// To view the first 20 lines of a use the 'n' argument.
head -n 20 someFile.txt
\end{lstlisting}

The tail command allows you to view the last section of content in
a file, this is useful for reading errors or the most recent output
from a program.

\begin{lstlisting}[language=make,showstringspaces=false,tabsize=4,frame=TB]
// View top contents of a file
tail someFile.txt

// To view the first 20 lines of a use the 'n' argument.
tail-n 20 someFile.txt
\end{lstlisting}

Finally, if you which to view the full file contents in manegable
sections the 'more' command breaks the file into sections to be viewed,
it waits for user input before continuing.

\begin{lstlisting}[language=make,showstringspaces=false,tabsize=4,frame=TB]
// View the full contents of a file in sections
more someFile.txt
\end{lstlisting}


\subsection*{Multiple Programs}

As mentioned, the command shell allows you to combine several commands
together via the pipe operator '\textbar{}'. For instance, when using
the history command there is often significant output that scrolls
down the screen. Using the pipe command, we can view the contents
of history command in a more manageable format.

\begin{lstlisting}[language=make,showstringspaces=false,tabsize=4,frame=TB]
// View the contents of history in sections.
history | more
\end{lstlisting}


\section*{Examples \label{sec:Examples}}

Command shells commands can be called directory from your C program,
consider the program provided in commandShell.c. This program takes
multiple arguments that are paths to directorys and lists the contents
of each directory.

\begin{lstlisting}[language=make,showstringspaces=false,tabsize=4,frame=TB]
#include<stdio.h>
#include<stdlib.h>
#include<string.h>

int main( int argc, char ** argv )
{
    for( int i = 1; i < argc; i++) {
            char path[256];
			// This function combines argument with the list command.
			// ex) ls someDirectory			
            snprintf(path, sizeof path, "%s%s", "ls  ", argv[i]);

			// This function calls the operating system with the string
			// provided in this case a list function
            system(path);
    }
}
\end{lstlisting}


\section*{Additional Information \label{sec:Additional-Information}}
\begin{itemize}
\item \url{http://linuxcommand.org/index.php} 
\item \url{http://www.cprogramming.com/tutorial/cfileio.html} 
\end{itemize}

\end{document}
