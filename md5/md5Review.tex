%% LyX 2.2.2 created this file.  For more info, see http://www.lyx.org/.
%% Do not edit unless you really know what you are doing.
\documentclass[12pt,english]{article}
\renewcommand{\familydefault}{\sfdefault}
\usepackage[latin9]{inputenc}
\usepackage{geometry}
\geometry{verbose,tmargin=1in,bmargin=1in,lmargin=1cm,rmargin=1in}
\usepackage{url}
\usepackage{amsmath}

\makeatletter
%%%%%%%%%%%%%%%%%%%%%%%%%%%%%% User specified LaTeX commands.
\date{ }



\usepackage{babel}
\usepackage{listings}
\renewcommand{\lstlistingname}{Listing}

\makeatother

\usepackage{babel}
\usepackage{listings}
\renewcommand{\lstlistingname}{Listing}

\begin{document}

\title{Review: MD5}
\maketitle

\section*{Overview}

MD5 is an algorithm used to create a 128-bit hash value. Initially,
MD5 was created for security purposes but it no longer is the most
secure way to encrypt information. However, it is an accessible introduction
to using encryption. This review covers a basic introduction to the
topic, including a ``Getting Started'' example, and some common
applications.

\section*{Learning Outcomes \label{sec:Learning-Outcomes}}

Upon successful completion of this review, you should be able to: 
\begin{itemize}
\item recognize the applications of MD5 
\item generate MD5 hashes 
\item use MD5 to validate data 
\end{itemize}

\section*{Introduction \label{sec:Introduction}}

MD5 was an algorithm originally created for security, however, the
algorithm has several vulnerabilities that make it no longer suitable
for protection. Now MD5 is mainly used for data validation and integrity,
via checksums. MD5 is a hash function that hash values of 128 bits.
MD5 is mainly a method of quickly evaluating transmitted data, they
check the data by using a hash function that creates a summation of
the binary data. Checksums do not 100\% guarantee the integrity of
the transmitted data is correct Although, it's very unlikely that
the hash value of corrupted data will match the hash value of a successfully
transferred file it can occur.

MD5 checksums known as md5sum, occurs on many servers, as an initial
step in validation.

\section*{Getting Started \label{sec:Getting-Started}}

To generate a md5sum for your text or data, in the terminal of a Linux
system, type: 
\begin{center}
\begin{lstlisting}[language=make,showstringspaces=false,tabsize=4,frame=TB]
echo -n "Your Text" | md5sum
\end{lstlisting}
\par\end{center}

This creates a md5 checksum of whatever text you've written if you
change the text a different hash value will be created. Echo -n is
an option that does not exclude the additional line break included
by echo. Then your text is piped into md5sum to create a hash value.
The expected MD5sum result should be provided to the server when downloading
and a match can be validated once the transfer is completed.

Creating a hash for a number of files to be transferred can be done
as follows: 
\begin{center}
\begin{lstlisting}[language=make,showstringspaces=false,tabsize=4,frame=TB]
md5sum filetohashA.txt filetohashB.txt filetohashC.txt > hash.md5
\end{lstlisting}
\par\end{center}

If you view the hash.md5 file you will see a similar 128-bit hash
value for each file. If you want to check the md5sum are correct,
use the -c option. 
\begin{center}
\begin{lstlisting}[language=make,showstringspaces=false,tabsize=4,frame=TB]
md5sum -c hash.md5
\end{lstlisting}
\par\end{center}

\section*{Examples \label{sec:Examples}}

In the examples folder, we provided two examples, md5simple.c calls
the md5 function using a system command on any arguments passed to
the program, note there is no error checking or handling for this
example. The 2nd example demonstrates how to incorporate the OpenSSL
library and call MD5.

\subsection*{Calling MD5 using System}

The core of the md5simple.c example is a for-loop which calls md5sum
on each filename passed in as an argument. The system function calls
a command as if it was input directly into the command line.

\begin{lstlisting}[language={C++},tabsize=4,frame=TB]
for( int i = 1; i < argc; i++) { 
	char file[256];
    snprintf(file, sizeof file, "%s%s", "md5sum ", argv[i]);
    system(file);
} 
\end{lstlisting}


\subsection*{Programming Example}

In order get started with using the MD5 algorithm, you have to install
the OpenSSL library. These libraries are available on the linux.socs.uoguelph.ca
server, other on Linux system you will have to call:

\begin{lstlisting}[language={C++},tabsize=4,frame=TB]
sudo apt-get install openssl libssl libssl-dev 
\end{lstlisting}

Once OpenSsl is installed, you can incorporate the library into your
c programs by compiling with a reference to the library. 

\begin{lstlisting}[language={C++},tabsize=4,frame=TB]
gcc yourFile.c -o yourProgramName -lcrypto 
\end{lstlisting}

The first requirement of programming using MD5 is including the proper
md5.h file. The core of the md5Programming example shown below uses the MD5 function
to convert a string into a hash value. 

The example uses a static string,
however, you could load this string with the contents of a file. It's
important to note that MD5 is 128 bits, a char is usually 8-bits and
the digest contains 16 values, 8{*}16 = 128 bits. The sprintf converts
each digest value into a 2 digit hexadecimal value for display. The
MD5 function is a shorthand for multiple steps in running the MD5
algorithm.  A slightly more complex example is given in the programming tools github repository (\url{https://github.com/judiMc/programming_tools})

\begin{lstlisting}[language={C++},tabsize=4,frame=TB]
#include <openssl/md5.h> 

unsigned char digest[MD5_DIGEST_LENGTH];
char string[] = "This works for a string, load and read a file could also work";
MD5((unsigned char*)&string, strlen(string), (unsigned char*)&digest);

// Convert the hash alue for displaying
char mdString[33];
for(int i = 0; i < 16; i++)          
	sprintf(&mdString[i*2], "%02x", (unsigned int)digest[i]);
printf("md5 digest: %s\n", mdString); 
\end{lstlisting}


\section*{Additional Information \label{sec:Additional-Information}}

The following links can provide useful additional information on MD5:

\url{http://www.makeuseof.com/tag/md5-hash-stuff-means-technology-explained/}

\url{https://en.wikipedia.org/wiki/MD5}

\url{https://www.go4expert.com/articles/md5-tutorial-t319/} 
\end{document}
