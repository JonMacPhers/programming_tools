%% LyX 2.2.2 created this file.  For more info, see http://www.lyx.org/.
%% Do not edit unless you really know what you are doing.
\documentclass[12pt,english]{article}
\renewcommand{\familydefault}{\sfdefault}
\usepackage[latin9]{inputenc}
\usepackage{geometry}
\geometry{verbose,tmargin=1in,bmargin=1in,lmargin=1cm,rmargin=1in}
\usepackage{url}
\usepackage{amsmath}

\makeatletter
%%%%%%%%%%%%%%%%%%%%%%%%%%%%%% User specified LaTeX commands.
\date{ }

\usepackage{babel}
\usepackage{listings}
\renewcommand{\lstlistingname}{Listing}

\usepackage{babel}
\usepackage{listings}
\renewcommand{\lstlistingname}{Listing}

\usepackage{babel}
\usepackage{listings}
\renewcommand{\lstlistingname}{Listing}

\makeatother

\usepackage{babel}
\usepackage{listings}
\renewcommand{\lstlistingname}{Listing}

\begin{document}

\title{Review: C Libraries}
\maketitle

\section*{Overview}

The command shell, command line, or shell is a user interface which
interacts directly with the operating system. Mastering fundamentals
of the command shell will be integral in developing your labs and
assignments throughout the course. We will discuss the most common
commands, identify the most common types of command shells, discuss
how to call system commands in a ``Getting Started'' example, and
additional resources.

\section*{Learning Outcomes \label{sec:Learning-Outcomes}}

Upon successful completion of this review, you should be able to:
\begin{itemize}
\item comprehend how to include a C-Library
\item differentiate between static and dynamic library types
\item develop and use your own static C-Library
\end{itemize}

\section*{Introduction \label{sec:Introduction}}

Software code libraries provide a convenient method for programmers
and projects to share common functionality. The C programming language
contains many libraries that can be included. By default, when you
use GCC the library libc.a is automatically included. Typically, libraries
files are prefixed with 'pre'. Two types of libraries exist, statically
linked ( {*}.a ) and dynamically linked ( {*}.so ), thus libc.a is
a statically linked library for the c language. On a Linux system,
standard libraries are typically stored in '/lib' or '/usr/lib'. Statically
linked libraries or static libraries are organized sets of software
code that can be used by other programsm once compiled. When a program or
executable compiles and links using a static library, a copy of the
static library is copied and included in the executable. If the code
in the static library needs to change, the entire executable must
be recompiled. This means the executable is a complete unit by itself
without external resources but is quite large in size.

Dynamically linked libraries (DLL) or dynamically linked shared object
libraries (.so) are libraries that are linked at runtime. Dynamically
linked libraries have the advantage of being able to change the library
or the main program without recompiling or linking the other parts
of the program. Dynamically linked libraries are more modular, allowing
programs to share the libraries and keep program sizes small. The
downside of dynamically linked and mutually shared libraries is that
the compatibility of the program and library can be out of sync. For
instance, if the dynamic library is updated to a newer version it
may become incompatible with the executable which will not be discovered
until that program is run. As static libraries errors are found at
compile time, dynamic libraries are found during run-time usually
via a program crash.

\section*{Getting Started \label{sec:Getting-Started}}

To get started with using C libraries, we will demonstrate how to
add one of the most common libraries, the math library. To include
a library for a command line compilation adds the '-l' option followed
by the name of the library.

\begin{lstlisting}[language=make,showstringspaces=false,tabsize=4,frame=TB]
# m is shortened name for the math library.
gcc main.c -lm -o myProgram
\end{lstlisting}

To find the system directory containing your .so or .a files use the
following GCC command.

\begin{lstlisting}[language=make,showstringspaces=false,tabsize=4,frame=TB]
# This will show the full path for the libc.a file
# other standard lib files will be at the same location.
gcc --print-file-name=libc.a
\end{lstlisting}


\section*{Static Library Example \label{sec:Examples}}

In this example, we will create a static library, the static library
only has one function to calculate the mean of two numbers. The files
myLib.h and myLib.c contain the contents of the static library.

\begin{lstlisting}[language=c,showstringspaces=false,tabsize=4,frame=TB]
// .h with one function taking 2 doubles
double mean(double, double);

// .c contains the implementation
double mean(double a, double b )
{
	return (a+b)/2.0;
}
\end{lstlisting}

Now we need to compile our static library files to ensure they work
correctly. Compile the two files and create an object file using the
following command:

\begin{lstlisting}[language=make,showstringspaces=false,tabsize=4,frame=TB]
gcc -c myLib.c -o myLib.o
\end{lstlisting}

This creates a single object file if your programming adds additional
files or makes files you could use all the objects files in your static
library. Next, we create the static library file using the ar command.

\begin{lstlisting}[language=make,showstringspaces=false,tabsize=4,frame=TB]
# ar command compresses the files into an archive
# You must include all required object files
# c option indicates to create lib if does not exist
# r option indicates to replace older *.o files

ar rc libmean.a myLib.o
\end{lstlisting}

As previously mentioned, a static library file contains all the files
required at compile time, so it must include every object related
to the library. Once the static library is created it needs to be
indexed so that the compiler can link and identify internal symbols.

\begin{lstlisting}[language=make,showstringspaces=false,tabsize=4,frame=TB]
# ranlib command indexes the lib file for compiler
ranlib libmean.a
\end{lstlisting}

Once you've created and indexed your library, you can use in other projects
and programs by including its directory in your compile statements
as follows. This GCC call includes setting path information for local
libraries using the '-L' option and indicating the name of the library
'-l' option. Once compiled, you can run your program as normal using
./meanMain. Additionally, if you change meanMain.c you don't need
to complete the previous steps for your library, only if you're library
changes do you need to recompile the library.

\begin{lstlisting}[language=make,showstringspaces=false,tabsize=4,frame=TB]
# ranlib command indexes the lib file for compiler
# -L indicates a path to the lib folder
# -l indicates the name of the library
# -l ignore the prefix 'lib'
gcc meanMain.c -L./lib -lmean.a -o meanMain
\end{lstlisting}


\section*{Additional Information \label{sec:Additional-Information}}
\begin{itemize}
\item \url{http://www.cs.dartmouth.edu/~campbell/cs50/buildlib.html}
\item \url{http://www.adp-gmbh.ch/cpp/gcc/create_lib.html}
\item \url{http://docencia.ac.upc.edu/FIB/USO/Bibliografia/unix-c-libraries.html#creating_static_archive}
\end{itemize}

\end{document}
